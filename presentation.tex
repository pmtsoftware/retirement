\documentclass{beamer}
\usepackage{polski}
\usepackage[polish]{babel}
\usepackage{array}
\usepackage{booktabs}
\usepackage{xcolor}

\usetheme{metropolis}
% \usecolortheme{spruce}

% \setbeamercolor{alerted text}{fg=red}
\setbeamertemplate{frame numbering}[fraction]

\title{Symulator emerytalny}
\subtitle{Symulator umożliwia obliczenie przybliżonej emerytury}

\author{Przemek Matuszny}
\date{05-10-2025}

\setbeamercovered{transparent=15}

\institute{Team: PMTS}

\begin{document}
    \maketitle

    \begin{frame}[t]{Parametry symulacji - użytkownika} \vspace{4pt}
        \begin{itemize}
            \item Wiek
            \item Płeć
            \item Wiek rozpoczęcia pracy
            \item Planowany wiek zakończenia pracy
            \item Wynagrodzenie
            \item Oczekiwana przez nas wartość emerytury
        \end{itemize}
    \end{frame}

    \begin{frame}[t]{Parametry symulacji - systemowe} \vspace{4pt}
        \begin{itemize}
            \item Wiek emerytalny kobiet/mężczyzn = 63/65
            \item Współczynnik składki ZUS = 17,2\%
            \item Współczynnik cpi waloryzacji składki = 1.05
            \item Średni czas życia = 76 lat
        \end{itemize}
        Parametrów systemowych nie można zmieniać - są stałe dla wszystkich symulacji.
    \end{frame}

    \begin{frame}[t]{Przebieg symulacji} \vspace{4pt}
        Przyszła emerytura wyliczana jest wg wzoru:

        \begin{equation}
            E_{n} = \sum_{i=1}^{n} \frac{S_i}{l}
        \end{equation}

        Gdzie:
        \begin{itemize}
            \item $ E_{n} $ - szacowana wielkość emerytury
            \item $ S_{i} $ - zwaloryzowana składka
            \item $ l $ - średnia długość życia
        \end{itemize}
        Dla uproszczenia brak podziału na konto główne i subkonto.
    \end{frame}

    \begin{frame}[t]{Aplikacja} \vspace{4pt}
        \begin{itemize}
            \item Aplikacja webowa - napisana w języku Haskell z wykorzystaniem bazy danych PostgreSQL
            \item Po zalogowaniu się użytkownik wprowadza parametry symulacji
            \item Po kliknięci przycisku \textit{Start} prezentowane jest okno z wynikami symulacji.
            \item Z poziomu tego okna możliwa jest zmiana parametrów i ponowne przeliczenie
            \item Dostępna jest też opcja wyświetlenia histori gdzie prezentowane są w formie tabeli wszystkie symulacje
        \end{itemize}
    \end{frame}

    \begin{frame}[t]{Demo} \vspace{4pt}

        Demo dostępne jest pod adresem \href{https://sim.hack.pmtsoftware.eu}{https://sim.hack.pmtsoftware.eu}
        Wymagane jest logowanie.
        \begin{itemize}
            \item użytkownik: \textit{demo@example.com}
            \item hasło: \textit{HackYeah2025}
        \end{itemize}
        Źródła do pobrania \href{https://github.com/pmtsoftware/retirement}{https://github.com/pmtsoftware/retirement}

    \end{frame}

    \begin{frame}[standout]
        Pytania?
    \end{frame}

\end{document}
